\documentclass[12pt,a4paper]{report}
\usepackage[french]{babel}
\usepackage[utf8]{inputenc}
\usepackage[T1]{fontenc}
\usepackage{graphicx}
\usepackage{hyperref}
\usepackage{listings}
\usepackage{xcolor}
\usepackage{geometry}
\usepackage{titlesec}
\usepackage{fancyhdr}

% Configuration de la page
\geometry{left=2.5cm,right=2.5cm,top=2.5cm,bottom=2.5cm}

% Configuration des en-têtes et pieds de page
\pagestyle{fancy}
\fancyhf{}
\fancyhead[L]{\leftmark}
\fancyhead[R]{\thepage}
\renewcommand{\headrulewidth}{0.4pt}

% Configuration des titres
\titleformat{\chapter}[display]
  {\normalfont\huge\bfseries}{\chaptertitlename\ \thechapter}{20pt}{\Huge}
\titlespacing*{\chapter}{0pt}{0pt}{20pt}

\begin{document}

% Page de garde
\begin{titlepage}
    \centering
    \vspace*{2cm}
    {\Huge\bfseries Rapport de Stage\par}
    \vspace{1.5cm}
    {\Large\itshape Développement d'une Application Web avec Next.js\par}
    \vspace{2cm}
    {\Large\itshape Nom et Prénom de l'étudiant\par}
    \vspace{1cm}
    {\large École d'Ingénieurs\par}
    \vspace{1cm}
    {\large Année Universitaire 2023-2024\par}
    \vfill
    {\large \today\par}
\end{titlepage}

% Table des matières
\tableofcontents
\newpage

% Remerciements
\chapter*{Remerciements}
Je tiens à remercier sincèrement toutes les personnes qui ont contribué à la réalisation de ce stage et à l'élaboration de ce rapport.

% Introduction
\chapter{Introduction}
\section{Contexte du stage}
Ce stage s'inscrit dans le cadre de la formation en ingénierie et vise à développer une application web moderne utilisant les technologies actuelles du développement web.

\section{Objectifs du stage}
Les principaux objectifs de ce stage sont :
\begin{itemize}
    \item Maîtriser le framework Next.js
    \item Développer une application web complète
    \item Appliquer les bonnes pratiques de développement
    \item Acquérir une expérience pratique en développement web
\end{itemize}

% Présentation du projet
\chapter{Présentation du Projet}
\section{Description du projet}
Le projet consiste en le développement d'une application web moderne utilisant Next.js, un framework React qui permet de créer des applications web performantes et évolutives.

\section{Technologies utilisées}
\begin{itemize}
    \item Next.js - Framework React
    \item React - Bibliothèque JavaScript
    \item TypeScript - Langage de programmation
    \item Tailwind CSS - Framework CSS
\end{itemize}

% Analyse des besoins
\chapter{Analyse des Besoins}
\section{Besoins Fonctionnels}
\subsection{Gestion des Utilisateurs}
\begin{itemize}
    \item Authentification et autorisation des utilisateurs
    \item Gestion des profils utilisateurs
    \item Système de rôles et permissions
\end{itemize}

\subsection{Gestion des Livraisons}
\begin{itemize}
    \item Suivi des livraisons en temps réel
    \item Gestion des commandes
    \item Système de notification
    \item Interface de suivi des livraisons
\end{itemize}

\subsection{Interface Utilisateur}
\begin{itemize}
    \item Interface responsive adaptée à tous les appareils
    \item Navigation intuitive
    \item Tableaux de bord personnalisés
    \item Système de recherche et filtrage
\end{itemize}

\section{Besoins Non Fonctionnels}
\subsection{Performance}
\begin{itemize}
    \item Temps de chargement optimisé
    \item Mise en cache efficace
    \item Optimisation des requêtes API
\end{itemize}

\subsection{Sécurité}
\begin{itemize}
    \item Protection contre les attaques CSRF
    \item Validation des données
    \item Gestion sécurisée des sessions
    \item Chiffrement des données sensibles
\end{itemize}

\subsection{Maintenabilité}
\begin{itemize}
    \item Code modulaire et réutilisable
    \item Documentation complète
    \item Tests automatisés
    \item Standards de codage respectés
\end{itemize}

% Architecture Technique
\chapter{Architecture Technique}
\section{Stack Technologique}
\subsection{Frontend}
\begin{itemize}
    \item Next.js 14 (Framework React)
    \item TypeScript
    \item Tailwind CSS pour le styling
    \item Shadcn/ui pour les composants
    \item React Query pour la gestion d'état
\end{itemize}

\subsection{Backend}
\begin{itemize}
    \item API RESTful
    \item Base de données relationnelle
    \item Système d'authentification JWT
    \item Middleware de validation
\end{itemize}

\section{Structure du Projet}
\subsection{Organisation des Dossiers}
\begin{itemize}
    \item \texttt{src/app} : Routes et pages de l'application
    \item \texttt{src/components} : Composants réutilisables
    \item \texttt{src/lib} : Utilitaires et configurations
    \item \texttt{src/hooks} : Hooks React personnalisés
    \item \texttt{src/utils} : Fonctions utilitaires
\end{itemize}

\section{APIs et Intégrations}
\subsection{APIs Internes}
\begin{itemize}
    \item API de gestion des utilisateurs
    \item API de gestion des livraisons
    \item API de notification
\end{itemize}

\subsection{APIs Externes}
\begin{itemize}
    \item Intégration Google Maps
    \item Services de géolocalisation
    \item Services de notification push
\end{itemize}

% Réalisation du projet
\chapter{Réalisation du Projet}
\section{Implémentation Technique}
\subsection{Architecture Frontend}
L'application utilise une architecture basée sur les composants React, avec :
\begin{itemize}
    \item Composants atomiques réutilisables
    \item Gestion d'état avec React Query
    \item Routing avec Next.js App Router
    \item Styling avec Tailwind CSS
\end{itemize}

\subsection{Sécurité et Performance}
\begin{itemize}
    \item Implémentation de l'authentification JWT
    \item Optimisation des images et assets
    \item Mise en cache des requêtes API
    \item Protection contre les attaques XSS et CSRF
\end{itemize}

\section{Fonctionnalités Développées}
\subsection{Module de Livraison}
\begin{itemize}
    \item Interface de suivi en temps réel
    \item Système de géolocalisation
    \item Gestion des statuts de livraison
    \item Notifications push
\end{itemize}

\subsection{Module Utilisateur}
\begin{itemize}
    \item Authentification sécurisée
    \item Gestion des profils
    \item Tableaux de bord personnalisés
    \item Historique des livraisons
\end{itemize}

% Conclusion
\chapter{Conclusion}
\section{Bilan du stage}
Ce stage a permis d'acquérir une expérience pratique significative dans le développement web moderne, en particulier avec Next.js et React.

\section{Perspectives}
Les compétences acquises pendant ce stage constituent une base solide pour de futurs projets de développement web et ouvrent la voie à l'exploration d'autres technologies modernes.

% Bibliographie
\chapter*{Bibliographie}
\begin{thebibliography}{9}
\bibitem{nextjs} Documentation officielle Next.js, \url{https://nextjs.org/docs}
\bibitem{react} Documentation officielle React, \url{https://reactjs.org/docs}
\bibitem{typescript} Documentation TypeScript, \url{https://www.typescriptlang.org/docs}
\end{thebibliography}

\end{document} 